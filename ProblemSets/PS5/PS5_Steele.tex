\documentclass{article}

% Language setting
% Replace `english' with e.g. `spanish' to change the document language
\usepackage[english]{babel}

% Set page size and margins
% Replace `letterpaper' with `a4paper' for UK/EU standard size
\usepackage[letterpaper,top=2cm,bottom=2cm,left=3cm,right=3cm,marginparwidth=1.75cm]{geometry}

% Useful packages
\usepackage{amsmath}
\usepackage{graphicx}
\usepackage[colorlinks=true, allcolors=blue]{hyperref}

\title{PS4}
\author{Colton Steele}
\date{February 23, 2023}
\begin{document}
\maketitle


\section{Problem #3}

The data that I scraped from a table was from NCAA softball's web page with statistics on individual batter data. I'm wanting to start doing different analysis and visualizations with softball data as I have not seen any prominent creators/analysts in the field doing softball analytics at a high level. I didn't use any tutorials to scrape this, it was pretty straightforward.

\section{Problem #4}

For this section, I used the the SoftballR package that was recently released by tmking02 (available here: https://github.com/tmking2002/softballR). This data is particularly interesting to me because in scraping the play-by-play data, I realized that the data contains pitch coordinates. An idea I had with this was to create heat charts for OU's softball team especially as we get further and further into this season. It would potentially even be more useful from a team perspective to know other teams heat charts than OU's own. There is also some data on hit coordinates, but a large number of null values so it wouldn't be as useful. The only resource I used to help use this package is the documentation available at the link I mentioned earlier.


\end{document}