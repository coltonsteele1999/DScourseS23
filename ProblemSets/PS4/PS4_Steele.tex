\documentclass{article}

% Language setting
% Replace `english' with e.g. `spanish' to change the document language
\usepackage[english]{babel}

% Set page size and margins
% Replace `letterpaper' with `a4paper' for UK/EU standard size
\usepackage[letterpaper,top=2cm,bottom=2cm,left=3cm,right=3cm,marginparwidth=1.75cm]{geometry}

% Useful packages
\usepackage{amsmath}
\usepackage{graphicx}
\usepackage[colorlinks=true, allcolors=blue]{hyperref}

\title{PS4}
\author{Colton Steele}
\date{February 23, 2023}
\begin{document}
\maketitle


\section{Introduction}

The data sources that I would like to scrape from are ones that house sports data. A website that I have scraped before is Pro Football Reference, and I could do a similar scraping for Pro Baseball Reference. I think doing sentiment analysis with Twitter data would be fun too. Additionally, as you know, I've done a lot of work with nflfastr and cfbfastr which scrape from various pages such as sports-reference.com, espn.com, nfl/ncaa.com etc. 

\section{Answers}
Problem 5
\\5d) The class of the object is character
\\Problem 6
\\6.7) df provides the output of tbl spark, tbl sql, tbl lazy, and tbl which means it's a spark table
\\df1 outputs tbl df, tbl, and data.frame which shows it is a data frame.
\\6.8) My column names in df have an underscore between the words while df1 has periods. My assumption is that in spark tables, you can't have periods between the words as periods can't be used in variable names while in R data frames, they can be.


\end{document}