\documentclass{article}

% Language setting
% Replace `english' with e.g. `spanish' to change the document language
\usepackage[english]{babel}

% Set page size and margins
% Replace `letterpaper' with `a4paper' for UK/EU standard size
\usepackage[letterpaper,top=2cm,bottom=2cm,left=3cm,right=3cm,marginparwidth=1.75cm]{geometry}

% Useful packages
\usepackage{amsmath}
\usepackage{graphicx}
\usepackage[colorlinks=true, allcolors=blue]{hyperref}

\title{PS1}
\author{Colton Steele}

\begin{document}
\maketitle

\begin{abstract}
\end{abstract}

My interests in data science and economics are very much oriented towards the sports world. After I wrote my paper for your econometrics class about going for it on 4th down in the NFL, I become very intrigued by the idea of being able to analyze data and better prove causation. I have spent a fair amount of free time testing out different projects using both college football and NFL data, primarily in the way of performance.

The first topic of interest for me is lookin at the invariance principle as it pertains to college football with both regular recruiting and transfer portal recruiting in the NIL era. The two big issues with that, sadly, are that the data on NIL is obviously not public, so some sort of estimation would have to occur, and that transfer portal data is sparse and I'm unaware of any centralized database that has tracked where each player has landed on top of transfer portal rankings still being in their infancy. The last idea I've had and that I'm interested in is doing a project on the shelf life of running backs in the NFL. Lately, it seems less and less teams are signing running backs to high dollar deals, and rarely drafting them in the first round while both of these things seemed more common not long ago. There seems to be a general consensus that these players' values have gone down, but I haven't found much research on the topic, at least ones that are available. 

I wanted to take this class for a few reasons. First, your class is what made me interested in this field and played a large role in me ultimately deciding to pursue my masters in sports data analytics. I learned a lot in that class and enjoyed it, despite it being one of the more intellectually challenging and difficult courses I took. It also fits into my graduate certificate while, in my estimation, will provide me with skills that will help me stand out in the labor force. 

For this class, I hope to diversify my range of skills so I'm at least familiar with many different tools and concepts that are used in data science. While I don't expect to become an expert in everything, having a base knowledge and understanding of a wide array of tools can provide more value towards a company down the road which would make me a more attractive candidate.

After graduation, I hope to find a job working for a sports franchise/organization either doing data science or business intelligence. I'm currently interning for the Thunder in their business intelligence department and have enjoyed it so far.




\section{Equation}

\[a^2+b^2=c^2\]

\end{document}
